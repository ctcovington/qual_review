\usepackage{ifthen}
\usepackage{mdwlist}
\usepackage{amsmath,amssymb,amsfonts,amsthm,bbm}
\usepackage{bm}
\usepackage[colorlinks=true,linkcolor=red,citecolor=blue,urlcolor=blue]{hyperref}
\usepackage{enumerate, enumitem}
\usepackage{graphicx}
\usepackage{xspace}
\usepackage{verbatim}
\usepackage{algorithm}
\usepackage[noend]{algpseudocode}
\usepackage[margin=0.8in]{geometry}
\usepackage{color}
\usepackage{thm-restate}
\usepackage{latexsym}
\usepackage{epsfig}
\usepackage{subcaption}
\usepackage{tabularx}
\usepackage{mathrsfs}
\usepackage{xcolor}
\usepackage{tcolorbox}
\usepackage{appendix}
\usepackage{titlesec}
\usepackage[colorinlistoftodos,textsize=scriptsize]{todonotes}
\usepackage[normalem]{ulem}
% \usepackage{enumerate}
\usepackage{natbib}
\usepackage{booktabs}
\definecolor{bgcolor}{RGB}{254, 252, 232}
\definecolor{OliveGreen}{rgb}{0,0.6,0}
\def\withcolors{1}
\def\withnotes{1}

\titlespacing\section{0pt}{12pt plus 4pt minus 2pt}{0pt plus 2pt minus 2pt}
\titlespacing\subsection{0pt}{12pt plus 4pt minus 2pt}{0pt plus 2pt minus 2pt}
\titlespacing\subsubsection{0pt}{12pt plus 4pt minus 2pt}{0pt plus 2pt minus 2pt}
\titlespacing\subsubsubsection{0pt}{12pt plus 4pt minus 2pt}{0pt plus 2pt minus 2pt}



\newtheorem{theorem}{Theorem}
\newtheorem{nontheorem}[theorem]{Non Theorem}
\newtheorem{proposition}[theorem]{Proposition}
\newtheorem{observation}[theorem]{Observation}
\newtheorem{remark}[theorem]{Remark}
\newtheorem{fact}[theorem]{Fact}
\newtheorem{lemma}[theorem]{Lemma}
\newtheorem{claim}[theorem]{Claim}
\newtheorem{corollary}[theorem]{Corollary}
\newtheorem{case}[theorem]{Case}
\newtheorem{dfn}[theorem]{Definition}
\newtheorem{definition}[theorem]{Definition}
\newtheorem{question}[theorem]{Question}
\newtheorem{openquestion}[theorem]{Open Question}
\newtheorem{res}[theorem]{Result}
\newtheorem{assumption}[theorem]{Assumption}
\newtheorem{example}[theorem]{Example}
\newtheorem{conjecture}[theorem]{Conjecture}

\numberwithin{theorem}{section} 
\numberwithin{nontheorem}{section} 
\numberwithin{proposition}{section} 
\numberwithin{observation}{section} 
\numberwithin{remark}{section} 
\numberwithin{fact}{section} 
\numberwithin{lemma}{section} 
\numberwithin{claim}{section} 
\numberwithin{corollary}{section} 
\numberwithin{case}{section} 
\numberwithin{dfn}{section} 
\numberwithin{definition}{section} 
\numberwithin{question}{section} 
\numberwithin{openquestion}{section} 
\numberwithin{res}{section} 
\numberwithin{assumption}{section}
\numberwithin{example}{section} 
\numberwithin{conjecture}{section} 

\newenvironment{proofsketch}{\noindent {\em {Proof (sketch):}}}{$\blacksquare$\vskip \belowdisplayskip}
\newenvironment{prevproof}[2]{\noindent {\em {Proof of {#1}~\ref{#2}:}}}{$\hfill\qed$\vskip \belowdisplayskip}

\ifnum\withcolors=1
  \newcommand{\new}[1]{{\color{red} {#1}}} % new
  \newcommand{\newer}[1]{{\color{blue} {#1}}} % even newer
  \newcommand{\newest}[1]{{\color{orange} {#1}}} % even even newer
  \newcommand{\newerest}[1]{{\color{blue!10!black!40!green} {#1}}} % you get the idea.
  \newcommand{\ccolor}[1]{{\color{OliveGreen}#1}} 
\else
  \newcommand{\new}[1]{{{#1}}}
  \newcommand{\newer}[1]{{{#1}}}
  \newcommand{\newest}[1]{{{#1}}}
  \newcommand{\newerest}[1]{{{#1}}}
  \newcommand{\ccolor}[1]{{#1}}
\fi

\ifnum\withnotes=1
  \newcommand{\cnote}[1]{\par\ccolor{\textbf{Christian: }\sf #1}} 
  \newcommand{\cfootnote}[1]{\footnote{{\bf \ccolor{Christian}}: {#1}}}

  \newcommand{\todonote}[2][]{\todo[size=\scriptsize,color=red!40,#1]{#2}}  
	\newcommand{\questionnote}[2][]{\todo[size=\tiny,color=blue!30]{#2}}
	\newcommand{\todonotedone}[2][]{\todo[size=\scriptsize,color=green!40]{$\checkmark$ #2}}
	\newcommand{\todonoteinline}[2][]{\todo[inline,size=\scriptsize,color=orange!40,#1]{#2}}  
  \newcommand{\marginnote}[1]{\todo[color=white,linecolor=black]{{#1}}}
\else
  \newcommand{\gnote}[1]{}
  \newcommand{\gfootnote}[1]{}
  \newcommand{\todonote}[2][]{\ignore{#2}}
	\newcommand{\questionnote}[2][]{\ignore{#2}}
	\newcommand{\todonotedone}[2][]{\ignore{#2}}
	\newcommand{\todonoteinline}[2][]{\ignore{#2}}
  \newcommand{\marginnote}[1]{\ignore{#1}}
\fi
\newcommand{\ignore}[1]{\leavevmode\unskip} % eat unnecessary spaces before
\newcommand{\gmargin}[1]{\questionnote{\gcolor{#1}}} 

% round parentheses
\newcommand{\paren}[1]{(#1)}
\newcommand{\Paren}[1]{\left(#1\right)}
\newcommand{\bigparen}[1]{\big(#1\big)}
\newcommand{\Bigparen}[1]{\Big(#1\Big)}
% square brackets
\newcommand{\brac}[1]{[#1]}
\newcommand{\Brac}[1]{\left[#1\right]}
\newcommand{\bigbrac}[1]{\big[#1\big]}
\newcommand{\Bigbrac}[1]{\Big[#1\Big]}
\newcommand{\Biggbrac}[1]{\Bigg[#1\Bigg]}
% floor
\newcommand{\floor}[1]{\lfloor#1\rfloor}
\newcommand{\Floor}[1]{\left\lfloor#1\right\rfloor]}
\newcommand{\bigfloor}[1]{\big\lfloor#1\big\rfloor}
\newcommand{\Bigfloor}[1]{\Big\lfloor#1\Big\rfloor}
\newcommand{\Biggfloor}[1]{\Bigg\lfloor#1\Bigg\rfloor}
% ceil
\newcommand{\ceil}[1]{\lceil#1\rceil}
\newcommand{\Ceil}[1]{\left\lceil#1\right\rceil}
\newcommand{\bigceil}[1]{\big\lceil#1\big\rceil}
\newcommand{\Bigceil}[1]{\Big\lceil#1\Big\rceil}
\newcommand{\Biggceil}[1]{\Bigg\lceil#1\Bigg\rceil}
% absolute value
\newcommand{\abs}[1]{\lvert#1\rvert}
\newcommand{\Abs}[1]{\left\lvert#1\right\rvert}
\newcommand{\bigabs}[1]{\big\lvert#1\big\rvert}
\newcommand{\Bigabs}[1]{\Big\lvert#1\Big\rvert}
% cardinality
\newcommand{\card}[1]{\lvert#1\rvert}
\newcommand{\Card}[1]{\left\lvert#1\right\rvert}
\newcommand{\bigcard}[1]{\big\lvert#1\big\rvert}
\newcommand{\Bigcard}[1]{\Big\lvert#1\Big\rvert}
% set
\newcommand{\set}[1]{\{#1\}}
\newcommand{\Set}[1]{\left\{#1\right\}}
\newcommand{\bigset}[1]{\big\{#1\big\}}
\newcommand{\Bigset}[1]{\Big\{#1\Big\}}
% norm
\newcommand{\norm}[1]{\lVert#1\rVert}
\newcommand{\Norm}[1]{\left\lVert#1\right\rVert}
\newcommand{\bignorm}[1]{\big\lVert#1\big\rVert}
\newcommand{\Bignorm}[1]{\Big\lVert#1\Big\rVert}
\newcommand{\Biggnorm}[1]{\Bigg\lVert#1\Bigg\rVert}
% 2-norm
\newcommand{\normt}[1]{\norm{#1}_2}
\newcommand{\Normt}[1]{\Norm{#1}_2}
\newcommand{\bignormt}[1]{\bignorm{#1}_2}
\newcommand{\Bignormt}[1]{\Bignorm{#1}_2}
% 2-norm squared
\newcommand{\snormt}[1]{\norm{#1}^2_2}
\newcommand{\Snormt}[1]{\Norm{#1}^2_2}
\newcommand{\bigsnormt}[1]{\bignorm{#1}^2_2}
\newcommand{\Bigsnormt}[1]{\Bignorm{#1}^2_2}
% norm squared
\newcommand{\snorm}[1]{\norm{#1}^2}
\newcommand{\Snorm}[1]{\Norm{#1}^2}
\newcommand{\bigsnorm}[1]{\bignorm{#1}^2}
\newcommand{\Bigsnorm}[1]{\Bignorm{#1}^2}
% 1-norm
\newcommand{\normo}[1]{\norm{#1}_1}
\newcommand{\Normo}[1]{\Norm{#1}_1}
\newcommand{\bignormo}[1]{\bignorm{#1}_1}
\newcommand{\Bignormo}[1]{\Bignorm{#1}_1}
% infty-norm
\newcommand{\normi}[1]{\norm{#1}_\infty}
\newcommand{\Normi}[1]{\Norm{#1}_\infty}
\newcommand{\bignormi}[1]{\bignorm{#1}_\infty}
\newcommand{\Bignormi}[1]{\Bignorm{#1}_\infty}
% inner product
\newcommand{\iprod}[1]{\langle#1\rangle}
\newcommand{\Iprod}[1]{\left\langle#1\right\rangle}
\newcommand{\bigiprod}[1]{\big\langle#1\big\rangle}
\newcommand{\Bigiprod}[1]{\Big\langle#1\Big\rangle}
%%% probability
% expectation, probability, variance
\newcommand{\Esymb}{\mathbb{E}}
\newcommand{\Psymb}{\mathbb{P}}
\newcommand{\Vsymb}{\mathbb{V}}
\DeclareMathOperator*{\E}{\Esymb}
\DeclareMathOperator*{\Var}{\textrm{Var}}
\DeclareMathOperator*{\Cov}{\textrm{Cov}}
\DeclareMathOperator*{\ProbOp}{\Psymb}
\renewcommand{\Pr}{\ProbOp}
%%% middle
%\newcommand{\given}{\;\middle\vert\;}
\newcommand{\given}{\mathrel{}\middle\vert\mathrel{}}
%\newcommand{\given}{\mathrel{}\middle|\mathrel{}}
% {{{ miscmacros }}}
% middle delimiter in the definition of a set
\newcommand{\suchthat}{\;\middle\vert\;}
% tensor product
\newcommand{\tensor}{\otimes}
% add explanations to math displays
\newcommand{\where}{\text{where}}
\newcommand{\textparen}[1]{\text{(#1)}}
\newcommand{\using}[1]{\textparen{using #1}}
\newcommand{\smallusing}[1]{\text{(\small using #1)}}
\newcommand{\by}[1]{\textparen{by #1}}
% spectral order (Loewner order)
\newcommand{\sge}{\succeq}
\newcommand{\sle}{\preceq}
% smallest and largest eigenvalue
\newcommand{\lmin}{\lambda_{\min}}
\newcommand{\lmax}{\lambda_{\max}}
\newcommand{\signs}{\set{1,-1}}
\newcommand{\varsigns}{\set{\pm 1}}
\newcommand{\maximize}{\mathop{\textrm{maximize}}}
\newcommand{\minimize}{\mathop{\textrm{minimize}}}
\newcommand{\subjectto}{\mathop{\textrm{subject to}}}
\renewcommand{\ij}{{ij}}
% symmetric difference
\newcommand{\symdiff}{\Delta}
\newcommand{\varsyff}{\bigtriangleup}
% set of bits
\newcommand{\bits}{\{0,1\}}
\newcommand{\sbits}{\{\pm1\}}
% no stupid bullets for itemize environmentx
% \renewcommand{\labelitemi}{--}
% control white space of list and display environments
\newcommand{\listoptions}{\labelsep0mm\topsep-0mm\itemindent-6mm\itemsep0mm}
\newcommand{\displayoptions}[1]{\abovedisplayshortskip#1mm\belowdisplayshortskip#1mm\abovedisplayskip#1mm\belowdisplayskip#1mm}
% short for emptyset
%\newcommand{\eset}{\emptyset}
% moved to mathabbreviations
% short for epsilon
%\newcommand{\e}{\epsilon}
% moved to mathabbreviations
% super index with parentheses
\newcommand{\super}[2]{#1^{\paren{#2}}}
\newcommand{\varsuper}[2]{#1^{\scriptscriptstyle\paren{#2}}}
% tensor power notation
\newcommand{\tensorpower}[2]{#1^{\tensor #2}}
% multiplicative inverse
\newcommand{\inv}[1]{{#1^{-1}}}
% dual element
\newcommand{\dual}[1]{{#1^*}}
% subset
%\newcommand{\sse}{\subseteq}
% moved to mathabbreviations
% vertical space in math formula
\newcommand{\vbig}{\vphantom{\bigoplus}}
% setminus
\newcommand{\sm}{\setminus}
% define something by an equation (display)
\newcommand{\defeq}{\stackrel{\mathrm{def}}=}
% define something by an equation (inline)
\newcommand{\seteq}{\mathrel{\mathop:}=}
% declare function f by $f \from X \to Y$
\newcommand{\from}{\colon}
% big middle separator (for conditioning probability spaces)
\newcommand{\bigmid}{~\big|~}
\newcommand{\Bigmid}{~\Big|~}
\newcommand{\Mid}{\nonscript\;\middle\vert\nonscript\;}
% better vector definition and some variations
%\renewcommand{\vec}[1]{{\bm{#1}}}
\newcommand{\bvec}[1]{\bar{\vec{#1}}}
\newcommand{\pvec}[1]{\vec{#1}'}
\newcommand{\tvec}[1]{{\tilde{\vec{#1}}}}
% punctuation at the end of a displayed formula
\newcommand{\mper}{\,.}
\newcommand{\mcom}{\,,}
% inner product for matrices
\newcommand\bdot\bullet
% transpose
\newcommand{\trsp}[1]{{#1}^\dagger}
% indicator function / vector
\DeclareMathOperator{\Ind}{\mathbf 1}
% place a qed symbol inside display formula
%\qedhere
% {{{ mathoperators }}}
\DeclareMathOperator{\Inf}{Inf}
\DeclareMathOperator{\nullity}{nullity}
\DeclareMathOperator{\Null}{null}
\DeclareMathOperator{\Tr}{Tr}
%\newcommand{\Tr}{\mathrm{Tr}}
\DeclareMathOperator{\SDP}{SDP}
\DeclareMathOperator{\sdp}{sdp}
\DeclareMathOperator{\val}{val}
\DeclareMathOperator{\LP}{LP}
\DeclareMathOperator{\OPT}{OPT}
\DeclareMathOperator{\opt}{opt}
\DeclareMathOperator{\vol}{vol}
\DeclareMathOperator{\poly}{poly}
\DeclareMathOperator{\qpoly}{qpoly}
\DeclareMathOperator{\qpolylog}{qpolylog}
\DeclareMathOperator{\qqpoly}{qqpoly}
\DeclareMathOperator{\argmax}{argmax}
\DeclareMathOperator{\polylog}{polylog}
\DeclareMathOperator{\supp}{supp}
\DeclareMathOperator{\dist}{dist}
\DeclareMathOperator{\sign}{sign}
\DeclareMathOperator{\conv}{conv}
\DeclareMathOperator{\Conv}{Conv}
\DeclareMathOperator{\rank}{rank}
\DeclareMathOperator{\diam}{diam}
% operators with limits
\DeclareMathOperator*{\median}{median}
\DeclareMathOperator*{\Median}{Median}
% smaller summation/product symbols
\DeclareMathOperator*{\varsum}{{\textstyle \sum}}
\DeclareMathOperator{\tsum}{{\textstyle \sum}}
\let\varprod\undefined
\DeclareMathOperator*{\varprod}{{\textstyle \prod}}
\DeclareMathOperator{\tprod}{{\textstyle \prod}}
% {{{ textabbreviations }}}
% some abbreviations
\newcommand{\ie}{i.e.,\xspace}
\newcommand{\eg}{e.g.,\xspace}
\newcommand{\Eg}{E.g.,\xspace}
\newcommand{\phd}{Ph.\,D.\xspace}
\newcommand{\msc}{M.\,S.\xspace}
\newcommand{\bsc}{B.\,S.\xspace}
\newcommand{\etal}{et al.\xspace}
\newcommand{\iid}{i.i.d.\xspace}
% {{{ foreignwords }}}
\newcommand\naive{na\"{\i}ve\xspace}
\newcommand\Naive{Na\"{\i}ve\xspace}
\newcommand\naively{na\"{\i}vely\xspace}
\newcommand\Naively{Na\"{\i}vely\xspace}
% {{{ names }}}
% Hungarian/Polish/East European names
\newcommand{\Erdos}{Erd\H{o}s\xspace}
\newcommand{\Renyi}{R\'enyi\xspace}
\newcommand{\Lovasz}{Lov\'asz\xspace}
\newcommand{\Juhasz}{Juh\'asz\xspace}
\newcommand{\Bollobas}{Bollob\'as\xspace}
\newcommand{\Furedi}{F\"uredi\xspace}
\newcommand{\Komlos}{Koml\'os\xspace}
\newcommand{\Luczak}{\L uczak\xspace}
\newcommand{\Kucera}{Ku\v{c}era\xspace}
\newcommand{\Szemeredi}{Szemer\'edi\xspace}
\newcommand{\Hastad}{H{\aa}stad\xspace}
\newcommand{\Hoelder}{H\"{o}lder\xspace}
\newcommand{\Holder}{\Hoelder}
\newcommand{\Brandao}{Brand\~ao\xspace}
% {{{ numbersets }}}
% number sets
\newcommand{\Z}{\mathbb Z}
\newcommand{\N}{\mathbb N}
\newcommand{\R}{\mathbb R}
\newcommand{\Q}{\mathbb Q}
%\newcommand{\C}{\mathbb C}
\newcommand{\Rnn}{\R_+}
\newcommand{\varR}{\Re}
\newcommand{\varRnn}{\varR_+}
\newcommand{\varvarRnn}{\R_{\ge 0}}
% {{{ problems }}}
% macros to denote computational problems
% use texorpdfstring to avoid problems with hyperref (can use problem
% macros also in headings
\newcommand{\problemmacro}[1]{\texorpdfstring{\textup{\textsc{#1}}}{#1}\xspace}
\newcommand{\pnum}[1]{{\footnotesize #1}}
% list of problems
\newcommand{\uniquegames}{\problemmacro{unique games}}
\newcommand{\maxcut}{\problemmacro{max cut}}
\newcommand{\multicut}{\problemmacro{multi cut}}
\newcommand{\vertexcover}{\problemmacro{vertex cover}}
\newcommand{\balancedseparator}{\problemmacro{balanced separator}}
\newcommand{\maxtwosat}{\problemmacro{max \pnum{3}-sat}}
\newcommand{\maxthreesat}{\problemmacro{max \pnum{3}-sat}}
\newcommand{\maxthreelin}{\problemmacro{max \pnum{3}-lin}}
\newcommand{\threesat}{\problemmacro{\pnum{3}-sat}}
\newcommand{\labelcover}{\problemmacro{label cover}}
\newcommand{\setcover}{\problemmacro{set cover}}
\newcommand{\maxksat}{\problemmacro{max $k$-sat}}
\newcommand{\mas}{\problemmacro{maximum acyclic subgraph}}
\newcommand{\kwaycut}{\problemmacro{$k$-way cut}}
\newcommand{\sparsestcut}{\problemmacro{sparsest cut}}
\newcommand{\betweenness}{\problemmacro{betweenness}}
\newcommand{\uniformsparsestcut}{\problemmacro{uniform sparsest cut}}
\newcommand{\grothendieckproblem}{\problemmacro{Grothendieck problem}}
\newcommand{\maxfoursat}{\problemmacro{max \pnum{4}-sat}}
\newcommand{\maxkcsp}{\problemmacro{max $k$-csp}}
\newcommand{\maxdicut}{\problemmacro{max dicut}}
\newcommand{\maxcutgain}{\problemmacro{max cut gain}}
\newcommand{\smallsetexpansion}{\problemmacro{small-set expansion}}
\newcommand{\minbisection}{\problemmacro{min bisection}}
\newcommand{\minimumlineararrangement}{\problemmacro{minimum linear arrangement}}
\newcommand{\maxtwolin}{\problemmacro{max \pnum{2}-lin}}
\newcommand{\gammamaxlin}{\problemmacro{$\Gamma$-max \pnum{2}-lin}}
\newcommand{\basicsdp}{\problemmacro{basic sdp}}
\newcommand{\dgames}{\problemmacro{$d$-to-1 games}}
\newcommand{\maxclique}{\problemmacro{max clique}}
\newcommand{\densestksubgraph}{\problemmacro{densest $k$-subgraph}}
% {{{ alphabet }}}
\newcommand{\cA}{\mathcal A}
\newcommand{\cB}{\mathcal B}
\newcommand{\cC}{\mathcal C}
\newcommand{\cD}{\mathcal D}
\newcommand{\cE}{\mathcal E}
\newcommand{\cF}{\mathcal F}
\newcommand{\cG}{\mathcal G}
\newcommand{\cH}{\mathcal H}
\newcommand{\cI}{\mathcal I}
\newcommand{\cJ}{\mathcal J}
\newcommand{\cK}{\mathcal K}
\newcommand{\cL}{\mathcal L}
\newcommand{\cM}{\mathcal M}
\newcommand{\cN}{\mathcal N}
\newcommand{\cO}{\mathcal O}
\newcommand{\cP}{\mathcal P}
\newcommand{\cQ}{\mathcal Q}
\newcommand{\cR}{\mathcal R}
\newcommand{\cS}{\mathcal S}
\newcommand{\cT}{\mathcal T}
\newcommand{\cU}{\mathcal U}
\newcommand{\cV}{\mathcal V}
\newcommand{\cW}{\mathcal W}
\newcommand{\cX}{\mathcal X}
\newcommand{\cY}{\mathcal Y}
\newcommand{\cZ}{\mathcal Z}

\newcommand{\bbA}{\mathbb A}
\newcommand{\bbB}{\mathbb B}
\newcommand{\bbC}{\mathbb C}
\newcommand{\bbD}{\mathbb D}
\newcommand{\bbE}{\mathbb E}
\newcommand{\bbF}{\mathbb F}
\newcommand{\bbG}{\mathbb G}
\newcommand{\bbH}{\mathbb H}
\newcommand{\bbI}{\mathbb I}
\newcommand{\bbJ}{\mathbb J}
\newcommand{\bbK}{\mathbb K}
\newcommand{\bbL}{\mathbb L}
\newcommand{\bbM}{\mathbb M}
\newcommand{\bbN}{\mathbb N}
\newcommand{\bbO}{\mathbb O}
\newcommand{\bbP}{\mathbb P}
\newcommand{\bbQ}{\mathbb Q}
\newcommand{\bbR}{\mathbb R}
\newcommand{\bbS}{\mathbb S}
\newcommand{\bbT}{\mathbb T}
\newcommand{\bbU}{\mathbb U}
\newcommand{\bbV}{\mathbb V}
\newcommand{\bbW}{\mathbb W}
\newcommand{\bbX}{\mathbb X}
\newcommand{\bbY}{\mathbb Y}
\newcommand{\bbZ}{\mathbb Z}

\newcommand{\scrA}{\mathscr A}
\newcommand{\scrB}{\mathscr B}
\newcommand{\scrC}{\mathscr C}
\newcommand{\scrD}{\mathscr D}
\newcommand{\scrE}{\mathscr E}
\newcommand{\scrF}{\mathscr F}
\newcommand{\scrG}{\mathscr G}
\newcommand{\scrH}{\mathscr H}
\newcommand{\scrI}{\mathscr I}
\newcommand{\scrJ}{\mathscr J}
\newcommand{\scrK}{\mathscr K}
\newcommand{\scrL}{\mathscr L}
\newcommand{\scrM}{\mathscr M}
\newcommand{\scrN}{\mathscr N}
\newcommand{\scrO}{\mathscr O}
\newcommand{\scrP}{\mathscr P}
\newcommand{\scrQ}{\mathscr Q}
\newcommand{\scrR}{\mathscr R}
\newcommand{\scrS}{\mathscr S}
\newcommand{\scrT}{\mathscr T}
\newcommand{\scrU}{\mathscr U}
\newcommand{\scrV}{\mathscr V}
\newcommand{\scrW}{\mathscr W}
\newcommand{\scrX}{\mathscr X}
\newcommand{\scrY}{\mathscr Y}
\newcommand{\scrZ}{\mathscr Z}
\newcommand{\sfE}{\mathsf E}
% {{{ leqslant }}}
% slanted lower/greater equal signs
\renewcommand{\leq}{\leqslant}
\renewcommand{\le}{\leqslant}
\renewcommand{\geq}{\geqslant}
\renewcommand{\ge}{\geqslant}
% {{{ varepsilon }}}
\let\epsilon=\varepsilon
%%% setup/equations
% {{{ restate }}}
% set of macros to deal with restating theorem environments (or anything
% else with a label)
% adapted from Boaz Barak
\newcommand\MYcurrentlabel{xxx}
% \MYstore{A}{B} assigns variable A value B
\newcommand{\MYstore}[2]{%
	\global\expandafter \def \csname MYMEMORY #1 \endcsname{#2}%
}
% \MYload{A} outputs value stored for variable A
\newcommand{\MYload}[1]{%
	\csname MYMEMORY #1 \endcsname%
}
% new label command, stores current label in \MYcurrentlabel
\newcommand{\MYnewlabel}[1]{%
	\renewcommand\MYcurrentlabel{#1}%
	\MYoldlabel{#1}%
}
% new label command that doesn't do anything
\newcommand{\MYdummylabel}[1]{}
\newcommand{\torestate}[1]{%
	% overwrite label command
	\let\MYoldlabel\label%
	\let\label\MYnewlabel%
	#1%
	\MYstore{\MYcurrentlabel}{#1}%
	% restore old label command
	\let\label\MYoldlabel%
}
\newcommand{\restatetheorem}[1]{%
	% overwrite label command with dummy
	\let\MYoldlabel\label
	\let\label\MYdummylabel
	\begin{theorem*}[Restatement of \cref{#1}]
		\MYload{#1}
	\end{theorem*}
	\let\label\MYoldlabel
}
\newcommand{\restatelemma}[1]{%
	% overwrite label command with dummy
	\let\MYoldlabel\label
	\let\label\MYdummylabel
	\begin{lemma*}[Restatement of \cref{#1}]
		\MYload{#1}
	\end{lemma*}
	\let\label\MYoldlabel
}
\newcommand{\restateprop}[1]{%
	% overwrite label command with dummy
	\let\MYoldlabel\label
	\let\label\MYdummylabel
	\begin{proposition*}[Restatement of \cref{#1}]
		\MYload{#1}
	\end{proposition*}
	\let\label\MYoldlabel
}
\newcommand{\restatefact}[1]{%
	% overwrite label command with dummy
	\let\MYoldlabel\label
	\let\label\MYdummylabel
	\begin{fact*}[Restatement of \cref{#1}]
		\MYload{#1}
	\end{fact*}
	\let\label\MYoldlabel
}
\newcommand{\restate}[1]{%
	% overwrite label command with dummy
	\let\MYoldlabel\label
	\let\label\MYdummylabel
	\MYload{#1}
	\let\label\MYoldlabel
}
% {{{ mathabbreviations }}}
\newcommand{\la}{\leftarrow}
\newcommand{\sse}{\subseteq}
\newcommand{\ra}{\rightarrow}
\newcommand{\e}{\epsilon}
\newcommand{\eps}{\epsilon}
\newcommand{\eset}{\emptyset}
% {{{ allowdisplaybreaks }}}
% allows page breaks in large display math formulas
\allowdisplaybreaks
% {{{ sloppy }}}
% avoid math spilling on margin
\sloppy
\newcommand*{\Id}{\mathrm{Id}}
\newcommand*{\grad}{\nabla}
\newcommand*{\Normop}[1]{\Norm{#1}}
\newcommand*{\normtv}[1]{\Norm{#1}_{\mathrm{TV}}
}
\newcommand*{\normf}[1]{\Norm{#1}_{\mathrm{F}}
}
\newcommand*{\normop}[1]{\norm{#1}_{\mathrm{op}}
}
\newcommand*{\st}[1]{{#1}^{\ast}
}
\newcommand*{\hatbetaLS}{\hat\beta_{\mathrm{LS}}
}
\newcommand*{\betahat}{\hat{\bm{\beta}}}
\newcommand*{\betazero}{\bm{\beta^0}}
\newcommand*{\etav}{\bm{\eta}}
\DeclareMathOperator{\diag}{diag}
\DeclareMathOperator{\support}{support}
\DeclareMathOperator{\Span}{span}
\DeclareMathOperator*{\argmin}{arg\,min}
\DeclareMathOperator{\subG}{subG}
% \newcommand*{\tran}{^{\mkern-1.5mu\mathsf{T}}}
\newcommand*{\transpose}[1]{{#1}{}^{\mkern-1.5mu\mathsf{T}}}
\newcommand*{\dyad}[1]{#1#1{}^{\mkern-1.5mu\mathsf{T}}}

\providecommand{\todo}{{\color{red}{\textbf{TODO}}}}
\providecommand{\Yij}{Y_{i,j}}
\providecommand{\yij}{y_{i,j}}
\providecommand{\wij}{w_{ij}}
\providecommand{\nulld}{\nu}
\providecommand{\planted}{\mu}
\providecommand{\Ep}{\E_{\planted}}
\providecommand{\En}{\E_{\nulld}}
\providecommand{\Hermitepolys}[1]{\cH_{\leq #1}}
\providecommand{\hermitepoly}[2]{H_{#2}\Paren{#1}}
\providecommand{\lowdegpolys}[1]{\R[Y]_{\leq#1}}
\providecommand{\multilinearpoly}[1]{\cM[Y]_{\leq#1}}
\providecommand{\Hermitemlpolys}[1]{\cH\cM_{\leq #1}}

\providecommand{\Tnote}{\Authornote{T}}
\providecommand{\Tcomment}{\Authorcomment{T}}
\providecommand{\Tfnote}{\Authorfnote{T}}

\newcommand{\cc}[1]{{\color{OliveGreen} Christian: #1}}

% double underline
\def\doubleunderline#1{\underline{\underline{#1}}}

% follows distribution as
\makeatletter
\newcommand{\distas}[1]{\mathbin{\overset{#1}{\kern\z@\sim}}}%
\newsavebox{\mybox}\newsavebox{\mysim}
\newcommand{\distras}[1]{%
	\savebox{\mybox}{\hbox{\kern3pt$\scriptstyle#1$\kern3pt}}%
	\savebox{\mysim}{\hbox{$\sim$}}%
	\mathbin{\overset{#1}{\kern\z@\resizebox{\wd\mybox}{\ht\mysim}{$\sim$}}}%
}
\makeatother

%%% constants
\newcommand{\rootconstant}{\kappa}
\newcommand{\ahat}{\hat{a}}
\newcommand{\balpha}{\bm{\alpha}}
\newcommand{\bbeta}{\bm{\beta}}
\newcommand{\bx}{\bm{x}}
\newcommand{\ba}{\bm{a}}
\newcommand{\bby}{\bm{y}}
\newcommand{\bw}{\bm{w}}
\newcommand{\bz}{\bm{z}}
\newcommand{\xhat}{\hat{x}}
\newcommand{\zhat}{\hat{z}}
\newcommand{\thetah}{\hat{\theta}}
\newcommand{\deltah}{\hat{\Delta}}
\newcommand{\thetas}{\theta^*}
\newcommand{\thetav}{\Norm{\theta}_{v\Paren{1}}}
\newcommand{\deltahv}{\Norm{\deltah}_{v\Paren{1}}}
\newcommand{\Normv}[1]{\Norm{#1}_{v\Paren{1}}}
\newcommand{\thetasv}{\Norm{\thetas}_{v\Paren{1}}}
\newcommand{\thetahv}{\Norm{\thetah}_{v\Paren{1}}}

\DeclareMathOperator{\SG}{SG}
\DeclareMathOperator{\SE}{SE}
\DeclareMathOperator{\perm}{perm}
\newcommand{\ball}[2]{\mathit{B}_{#2}\left( #1 \right)}
\newcommand{\wh}[1]{\ensuremath{\widehat{#1}}}
\newcommand{\wt}[1]{\ensuremath{\widetilde{#1}}}
\newcommand{\wb}[1]{\ensuremath{\overline{#1}}}

\def\*#1{\mathbf{#1}}

% define highlighter for algorithmicx
\def\HiLi{\leavevmode\rlap{\hbox to \hsize{\color{yellow!50}\leaders\hrule height 1.2 \baselineskip depth .5ex\hfill}}}

% CoinPress 
\newcommand{\mvmeaniter}{\textsc{MVMRec}}
\newcommand{\mvmeanstep}{\textsc{MVM}}
\newcommand{\mvcoviter}{\textsc{MVCRec}}
\newcommand{\mvcovstep}{\textsc{MVC}}
\newcommand{\hpub}{\textsc{HPUB}}
\newcommand{\approxub}{\textsc{ApproxUB}}
\newcommand{\seub}{\textsc{SEUB}}
\newcommand{\createthetahat}{\textsc{CreateEstimator}}
\newcommand{\covcomb}{\textsc{CovarianceCombination}}
\newcommand{\psdproj}{\textsc{PSDProjection}}
\newcommand{\conseig}{\textsc{ConsistentEigenvectors}}
\newcommand{\cisimulation}{\textsc{ConfidenceIntervalSimulation}}
\newcommand{\gvdp}{\textsc{GVDP}}
\newcommand{\blb}{\textsc{BLB}}
\newcommand{\abovethreshold}{\textsc{AboveThreshold}}

% Christian's new stuff
\definecolor{dred}{rgb}{0.75, 0.0, 0.2}
\definecolor{dgreen}{rgb}{0.33, 0.42, 0.18}
\definecolor{dblue}{rgb}{0.0, 0.2, 0.4}
\definecolor{crimson}{rgb}{0.86, 0.08, 0.24}
\newcommand{\comm}[1]{\hspace{5pt}\color{crimson}\left(\text{#1}\right)}
\newcommand{\1}[1]{\mathbbm{1}(#1)}
\let\ss\subsection
\let\sss\subsubsection
\let\ssss\subsubsubsection
\newcommand{\cv}{\rightarrow}
\newcommand{\cvd}{\overset{d}{\rightarrow}}
\newcommand{\cvPr}{\overset{\Pr}{\rightarrow}}
\newcommand{\cvas}{\overset{a.s.}{\rightarrow}}
\newcommand{\cvL}[1]{\overset{L^{#1}}{\rightarrow}}
\newcommand{\boxnote}[1]{
	\begin{tcolorbox}[colback=red!5!white,colframe=red!75!black]
	#1
	\end{tcolorbox}
}